\documentclass[a4paper]{article}
\usepackage[utf8]{inputenc}
\usepackage[T1]{fontenc}
\usepackage[light,condensed,math]{kurier}
\usepackage[ngerman]{babel}
\usepackage{ntheorem}
\usepackage{graphicx}
\usepackage{floatrow}
\usepackage{float}
\usepackage{hyperref}

\theoremstyle{break}
\newtheorem{defi}{Definition}[section]
\newtheorem{ex}{Beispiel}[section]
\newtheorem{ann}{Bemerkung}[section]
\newtheorem{why}{Vorteile}[section]
\newtheorem{whynot}{Nachteile}[section]

\title{SWT 1: Vorgehensmodelle}
\author{Sebastian Markgraf}

\begin{document}
    \maketitle
    \tableofcontents
    \newpage

    \section{Wasserfall}
    \begin{defi}
      Das \textbf{Wasserfallmodell} ist ein lineares (nicht iteratives) Vorgehensmodell, das insbesondere für die Softwareentwicklung verwendet wird und das in aufeinander folgenden Projektphasen organisiert ist. Dabei gehen die Phasenergebnisse wie bei einem Wasserfall immer als bindende Vorgaben für die nächsttiefere Phase ein.
    \end{defi}
    
    
    \section{V-Modell}
      \begin{defi}
        Das \textbf{V-Modell} ist ein Vorgehensmodell in der Softwareentwicklung, bei dem der Softwareentwicklungsprozess in Phasen organisiert wird. Neben diesen Entwicklungsphasen definiert das V-Modell auch das Vorgehen zur Qualitätssicherung (Testen) phasenweise. Auf der linken Seite wird mit einer funktionalen/fachlichen Spezifikation begonnen, die immer tiefer detailliert zu einer technischen Spezifikation und Implementierungsgrundlage ausgebaut wird. In der Spitze erfolgt die Implementierung, die anschließend auf der rechten Seite gegen die entsprechenden Spezifikationen der linken Seite getestet wird.
      \end{defi}
    
    \section{Agile Verfahren}
        \subsection{Agiles Manifest}
        \begin{flushleft}
          Wir erschließen bessere Wege, Software zu entwickeln,
          indem wir es selbst tun und anderen dabei helfen.
          Durch diese Tätigkeit haben wir diese Werte zu schätzen gelernt:
          
          Individuen und Interaktionen mehr als Prozesse und Werkzeuge
          Funktionierende Software mehr als umfassende Dokumentation
          Zusammenarbeit mit dem Kunden mehr als Vertragsverhandlung
          Reagieren auf Veränderung mehr als das Befolgen eines Plans

          Das heißt, obwohl wir die Werte auf der rechten Seite wichtig finden,
          schätzen wir die Werte auf der linken Seite höher ein. \footnote{\url{http://agilemanifesto.org/}}.
        \end{flushleft}

        \subsection{Eigenschaften}
        \begin{itemize}
            \item Minimum an Vorausplanung
            \item Planung erfolgt inkrementell
            \item Vemeiden unterstützender Dokumente
            \item Einbeziehung des Kunden in die Entwicklung
        \end{itemize}
        
          
    \section{Extreme Programming XP}
    \begin{defi}
      \textbf{Extreme Programming (XP, auch Extremprogrammierung)} ist eine Methode, die das Lösen einer Programmieraufgabe in den Vordergrund der Softwareentwicklung stellt und dabei einem formalisierten Vorgehen geringere Bedeutung zumisst.
    \end{defi}

    \subsection{Praktiken}
    
    \section{Scrum}
    \begin{defi}
      Scrum ist die Bezeichnung für ein Vorgehensmodell des Projekt- und Produktmanagements, insbesondere zur agilen Softwareentwicklung. Es wurde ursprünglich in der Softwaretechnik entwickelt, ist aber davon unabhängig.
    \end{defi}

    \subsection{Die 3 Säulen}
    
    


\end{document}
