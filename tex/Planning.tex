\documentclass[a4paper]{article}
\usepackage{etoc}
\usepackage[utf8]{inputenc}
\usepackage[T1]{fontenc}
\usepackage[light,condensed,math]{kurier}
\usepackage[ngerman]{babel}
\usepackage{ntheorem}
\usepackage{graphicx}
\usepackage{floatrow}
\usepackage{float}
\usepackage{hyperref}


\theoremstyle{break}
\newtheorem{defi}{Definition}[section]
\newtheorem{ex}{Beispiel}[section]
\newtheorem{why}{Vorteile}[section]
\newtheorem{whynot}{Nachteile}[section]
\title{SWT 1: Planung}
\author{Adrian E. Lehmann}

\begin{document}
	\maketitle
	\tableofcontents
	\newpage

	
\section{Anforderungserhebung}
	\textit{engl. Requirements elicitation}
\subsection{Methoden der Anforderungserhebung}	
\subsubsection{Fragebögen}
	\begin{why}
		\begin{enumerate}
			\item Kostengünstig
			\item Deckt viele Leute ab (gut für große Projekte)
			\item Relativ schnell
		\end{enumerate}
	\end{why}
	\begin{whynot}
		\begin{enumerate}
			\item Unpräzise
			\item Deckt evtl. nicht die eigentlichen Probleme ab
			\item Niedrige Motivation zum Ausfüllen unter Mitarbeitern
		\end{enumerate}
	\end{whynot}
\subsubsection{Interviews}
	\begin{why}
		\begin{enumerate}
			\item Sehr präzise Einsicht in Anwenderwünsche
		\end{enumerate}
	\end{why}
	\begin{whynot}
		\begin{enumerate}
			\item Deckt wenige Leute ab
			\item Teuer
			\end{enumerate}
	\end{whynot}
\subsubsection{Dokumenten- \& Aufgabenanalyse}
	\begin{why}
		\begin{enumerate}
			\item Direkte Einsicht in Abläufe
			\item Objektive Eindrücke
		\end{enumerate}
	\end{why}
	\begin{whynot}
		\begin{enumerate}
			\item Geht nicht auf Nutzerwünsche ein
			\item u. U. aufwendig
		\end{enumerate}
	\end{whynot}	
\subsubsection{Szenarien}
	\begin{why}
		\begin{enumerate}
			\item Direkte Einsicht in Abläufe
			\item Objektive Eindrücke
		\end{enumerate}
	\end{why}
	\begin{whynot}
		\begin{enumerate}
			\item Geht nicht auf Nutzerwünsche ein
			\item u. U. aufwendig
		\end{enumerate}
	\end{whynot}
\subsection{Anwendungsfälle}
\subsection{Typen der Anforderungen}
\subsubsection{Funktionale Anforderungen}
\subsubsection{Nicht-funktionale Anforderungen}
\subsubsection{Einschränkungen}
\subsection{Validierung von Anforderungen}
\newpage
\section{Lastenheft}
\subsection{Gliederung}
\localtableofcontents
\subsubsection{Zielbestimmung}
\subsubsection{Produkteinsatz}
\subsubsection{Funktionale Anforderungen}
\subsubsection{Produktdaten}
\subsubsection{Nichtfunktionale Anforderungen}
\subsubsection{Systemmodelle}
\subsubsection{Glossar}
\newpage
\section{Durchführbarkeitsuntersuchung}
\end{document}