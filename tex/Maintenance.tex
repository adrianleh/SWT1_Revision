\documentclass[a4paper]{article}
\usepackage[utf8]{inputenc}
\usepackage[T1]{fontenc}
\usepackage[light,condensed,math]{kurier}
\usepackage[ngerman]{babel}
\usepackage{ntheorem}
\usepackage{graphicx}
\usepackage{floatrow}
\usepackage{float}
\usepackage{hyperref}

\theoremstyle{break}
\newtheorem{defi}{Definition}[section]
\newtheorem{ex}{Beispiel}[section]
\newtheorem{why}{Vorteile}[section]
\newtheorem{whynot}{Nachteile}[section]
\title{SWT 1: Die Abnahme-, Einführungs-, Wartungs- \& Pflegephase}
\author{Adrian E. Lehmann}

\begin{document}
	\maketitle
	\tableofcontents
	\newpage

\section{Abnahmephase}
Die Tätigkeiten der \textbf{Abnahmephase} umfassen:
\begin{enumerate}
	\item Übergabe des Gesamtprodukts einschließlich der gesamten Dokumentation an den Auftraggeber
	\item Ein Abnahmetest
	\item (Meistens) Belastungs- oder Stresstests
\end{enumerate}
Das Ergebnis der Abnahmephase ist ein \textbf{Abnahmeprotokoll}
\begin{defi}[Abnahme]
	Die formale \textbf{Abnahme} ist die (schriftliche) Erklärung der Annahme eines Produkts durch den Auftraggeber (im juristischen Sinne)
\end{defi}
	
\subsection{Abnahmetest}
Im Abnahmetest werden folgende Qualitätsmerkmale folgendes geprüft:\newline
\begin{enumerate}
	\item Merkmale zur Produktnutzung
		\subitem Nutzbarkeit
		\subitem Integrität
		\subitem Effizienz
		\subitem Korrektheit 
		\subitem Zuverlässigkeit
	\item Merkmale zu Wartung \& Pflege
		\subitem Wartbarkeit
		\subitem Testbarkeit
		\subitem Flexibilität
\end{enumerate}
\section{Einführungsphase}
Die Tätigkeiten der \textbf{Einführungsphase} umfassen:
\begin{enumerate}
	\item Installation des Produktes
	\item Schulung der Benutzer und des Betriebspersonals
	\item Inbetriebnahme des Produkts
\end{enumerate}
Das Ergebnis der Einführungsphase ist ein \textbf{Einführungsprotokoll}
\subsection{Umstellung}
Bei der Einführung eines neuen Systems ist es wichtig eine \textbf{Umstellung} vom alten auf das neue System vorzunehmen. Hierbei vor allem müssen alte Datenbestände auf das neue System angepasst werden (Falls das alte System nicht digital ist, ist der Aufwand hierbei sehr hoch). \newline Weiterhin müssen u.U. Konvertierungsprogramme erstellt werden und es muss geprüft werden, dass die Datenbestände weiterhin korrekt sind.
\subsection{Inbetriebnahme}
Es gibt 3 Arten der \textbf{Inbetriebnahme}:
\subsubsection{Direkte Umstellung}
	\begin{defi}
		Bei der \textbf{direkten Umstellung} wird das alte System unmittelbar und vollständig durch das neue ersetzt
	\end{defi}
	\begin{why}
		\begin{enumerate}
			\item Günstig
			\item Sofort alle Vorteile des neuen Systems verfügbar
		\end{enumerate}
	\end{why}
	\begin{whynot}
		\begin{enumerate}
			\item Risikoreich
		\end{enumerate}
	\end{whynot}
\subsubsection{Parallellauf}
	\begin{defi}
		Beim \textbf{Parallellauf} werden beide Systeme (das neue \& das alte) simultan eingesetzt.
	\end{defi}
	\begin{why}
		\begin{enumerate}
			\item Sicherheit, falls das neue System fehlschlägt
			\item Sofort alle Vorteile des neuen Systems verfügbar
		\end{enumerate}
	\end{why}
	\begin{whynot}
		\begin{enumerate}
			\item Teuer \& aufwendig
			\item Redundanz
			\item System müssen teilweise eingeschränkt werden
			\item Evtl. bleiben Vorbehalte gegenüber des neuen Systems bestehen (z.B. "Ich will nicht wechseln, Windoof XP\footnote{Der Name dieser Software wurde in keinerlei Relationen zu jeglicher bestehender Software, sonder rein arbiträr gewählt} funktioniert immer noch super")
		\end{enumerate}
	\end{whynot}
\subsubsection{Versuchslauf}
\setcounter{secnumdepth}{4} % So I can use paragraph to wrap (<- see I'm using design patterns) \subsubsubsection - which for some unknown reason does not exist -.-
Hierbei gibt es zwei zu betrachtende Möglichkeiten:
\paragraph{1. Methode: Prüflauf}
	\begin{defi}
		Beim \textbf{Versuchslauf} dieser Methode arbeitet das neue System mit Daten aus vergangenen Perioden, so dass die Ergebnisse bekannt sind und überprüft werden können
	\end{defi}
	\begin{why}
		\begin{enumerate}
			\item Test unter realen Bedingungen
			\item Sicherheit
		\end{enumerate}
	\end{why}
	\begin{whynot}
		\begin{enumerate}
			\item Teuer \& aufwendig
			\item Vorteile des neuen System können nicht genossen werden
		\end{enumerate}
	\end{whynot}
\paragraph{2. Methode: Piloteinführung}
\begin{defi}
	Beim \textbf{Versuchslauf} dieser Methode werden Teile des Systems auf das neue umgestellt während andere Teile beim alten Belassen werden
\end{defi}
\begin{why}
	\begin{enumerate}
		\item Bessere Schulungsmöglichkeiten
		\item Kein Totalausfall möglich (bzw. unwahrscheinlicher, aber immer noch mögl.)
		\item Testen zuerst in unwichtigen Bereichen (Looking at you, WiWis)
	\end{enumerate}
\end{why}
\begin{whynot}
	\begin{enumerate}
		\item Teuer \& aufwendig
		\item Vorteile des neuen System können nicht von allen genossen werden
		\item u. U. fühlen sich Teile der Firma mit dem alten System benachteiligt (oder schlimmer: bevorteiligt)
	\end{enumerate}
\end{whynot}
\subsection{Einführung auf dem anonymen Markt: Pilotphase}
\begin{defi}
Bei dieser Einführung werden einzelne Personen aus der Zielgruppe der Software ausgewählt und es werden für diese Pilotversionen (Beta-Versionen) bereit gestellt.
\end{defi}
\begin{why}
	\begin{enumerate}
		\item Direkte Feedbackmöglichkeit
		\item Kein Totalausfall möglich (bzw. unwahrscheinlicher, aber immer noch mögl.)
	\end{enumerate}
\end{why}
\begin{whynot}
	\begin{enumerate}
		\item Takes time \& my Code always works before testing\footnote{Dies ist kein Nachteil, sondern eine sarkastische Darstellung, dass es keine echten Nachteile gibt}
	\end{enumerate}
\end{whynot}
\section{Wartung \& Pflege}
Beginnt nach der \emph{erfolgreichen} Abnahme und Einführung des Produktes, denn:
\begin{itemize}
\item das Produkt kann im Alltag versagen
\item die Umweltbedingungen verändern sich(Systemsoftware/Hardware)
\item neue Wünsche oder Anforderungen
\end{itemize}

Falls diese Punkte nicht behoben werden, dann \emph{altert} die Software und kann damit nicht mehr für den vorgesehenen Zweck eingesetzt werden.

Der Aufwand für Wartung und Pflege ist meist signifikant höher als der Aufwand bei der Entwicklung.

\subsection{Wartung}
Suchen und beheben von Defekten die im Betrieb auftreten, daher nicht planbar.

\subsection{Pflege}
Durchführung von Anpassungen, Änderungen und Erweiterungen, die im Betrieb festgestellt werden. Daher planbar.

Da die Pflege einer Weiterentwicklung entspricht ist es sinnvoll den normalen Software-Entwicklungsprozess für die Aktivitäten der Pflege einzusetzen.

Der Aufwand steigt mit dem Altern der Software und dem Umfang der Software(10\% pro Jahr). Daher muss irgendwann entschieden werden ob die Software weiter gepflegt wird, saniert werden soll oder durch eine neue ersetzt wird.

\subsection{Kategorien der Wartung \& Pflege}

\begin{itemize}
\item korrektive Tätigkeiten(Wartung)
  \begin{enumerate}
  \item Stabilisierung/Korrektur\\
    Die Behebung der Defekte, die seit Entwicklung oder erst mit der Wartung in das Produkt gekommen sind.

    Viele Defekte werden erst im Einsatz entdeckt und die meisten Defekte entstehen erst bei der Wartung(beispielweise durch schlechte Konstruktion, schlechte Dokumentation oder fehlendes Produktverständnis durch das Wartungspersonal)
  \item Optimierung/Leistungsverbesserung\\
    Software wird meistens mit der Funktionsfähigkeit freigegeben und dadurch wird selten vor der Freigabe optimiert. Daher verwendet Software meist mehr Zeit oder Speicher als zur Verfügung steht.

  \end{enumerate}
\item progressive Tätigkeiten(Pflege)
  \begin{enumerate}
    \setcounter{enumi}{2}
  \item Anpassung/Änderung\\
    Durch Umwelt erzwungene Änderungen(technisch/GUI/Funktionen).
  \item Erweiterung\\
    Erweiterung von Funktionalität die sich durch nicht implementierte Funktionen während der Erstentwicklung oder durch neue Anforderungen ergeben.
  \end{enumerate}

  \subsubsection{Alternative Klassifikation}
  \begin{enumerate}
  \item Korrigierende Aktivitäten
  \item Anpassende Aktivitäten
  \item Perfektionierende Aktivitäten
  \end{enumerate}
  \noindent Am meisten Zeit wird in Anpassungen und Erweiterungen investiert.
\end{itemize}

\subsection{Software-Sanierung}
Verstehen der alten Software und Umwandlung in eine bessere wartbare Form um schlie\ss{}lich Änderungen vorzunehmen.


\section{Änderungsverwaltung}
Erfassung von Änderungsvorschlägen und Fehlermeldungen. Danach die Entscheidungen über das Bearbeiten:
\begin{enumerate}
\item Annahme/Ablehnung
\item Wählen eines Lösungsvorschlags
\item technische und zeitliche Auswirkungen
\item bearbeiten veranlassen
\item bündeln der Änderungen
\end{enumerate}

Bei Fehlermeldungen:
\begin{enumerate}
\item müssen reproduzierbar sein.
\item alle relevanten Informationen werden vom Benutzer zusammengefasst(Eingaben, Produktversion, Systemsoftware).
\end{enumerate}

Eine Änderungsverfolgung(bug tracker) hilft damit dass sich mehrere Benutzer beziehungsweise Entwickler austauschen können. Die Meldung kann auch automatisiert erfolgen, dabei muss jedoch eventuell Datenschutzrechtliche Probleme bedacht werden.

Beispiele für bug tracker:
\begin{enumerate}
\item Bugzilla
\item JIRA
\item Trac
\end{enumerate}

\end{document}