\documentclass[a4paper]{article}
\usepackage[utf8]{inputenc}
\usepackage[T1]{fontenc}
\usepackage[light,condensed,math]{kurier}
\usepackage[ngerman]{babel}
\usepackage{ntheorem}
\usepackage{graphicx}
\usepackage{floatrow}
\usepackage{float}
\usepackage{hyperref}

\theoremstyle{break}
\newtheorem{defi}{Definition}[section]
\newtheorem{ex}{Beispiel}[section]
\newtheorem{why}{Vorteile}[section]
\newtheorem{whynot}{Nachteile}[section]
\title{SWT 1: Die Abnahme-, Einführungs-, Wartungs- \& Pflegephase}
\author{Adrian E. Lehmann}

\begin{document}
	\maketitle
	\tableofcontents
	\newpage

\section{Abnahmephase}
Die Tätigkeiten der \textbf{Abnahmephase} umfassen:
\begin{enumerate}
	\item Übergabe des Gesamtprodukts einschließlich der gesamten Dokumentation an den Auftraggeber
	\item Ein Abnahmetest
	\item (Meistens) Belastungs- oder Stresstests
\end{enumerate}
Das Ergebnis der Abnahmephase ist ein \textbf{Abnahmeprotokoll}
\begin{defi}[Abnahme]
	Die formale \textbf{Abnahme} ist die (schriftliche) Erklärung der Annahme eines Produkts durch den Auftraggeber (im juristischen Sinne)
\end{defi}
	
\subsection{Abnahmetest}
Im Abnahmetest werden folgende Qualitätsmerkmale folgendes geprüft:\newline
\begin{enumerate}
	\item Merkmale zur Produktnutzung
		\subitem Nutzbarkeit
		\subitem Integrität
		\subitem Effizienz
		\subitem Korrektheit 
		\subitem Zuverlässigkeit
	\item Merkmale zu Wartung \& Pflege
		\subitem Wartbarkeit
		\subitem Testbarkeit
		\subitem Flexibilität
\end{enumerate}
\section{Einführungsphase}

\section{Wartung \& Pflege}

\end{document}